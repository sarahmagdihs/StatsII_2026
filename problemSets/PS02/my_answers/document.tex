\documentclass[12pt,letterpaper]{article}
\usepackage{graphicx,textcomp}
\usepackage{natbib}
\usepackage{setspace}
\usepackage{fullpage}
\usepackage{color}
\usepackage[reqno]{amsmath}
\usepackage{amsthm}
\usepackage{fancyvrb}
\usepackage{amssymb,enumerate}
\usepackage[all]{xy}
\usepackage{endnotes}
\usepackage{lscape}
\newtheorem{com}{Comment}
\usepackage{float}
\usepackage{hyperref}
\newtheorem{lem} {Lemma}
\newtheorem{prop}{Proposition}
\newtheorem{thm}{Theorem}
\newtheorem{defn}{Definition}
\newtheorem{cor}{Corollary}
\newtheorem{obs}{Observation}
\usepackage[compact]{titlesec}
\usepackage{dcolumn}
\usepackage{tikz}
\usetikzlibrary{arrows}
\usepackage{multirow}
\usepackage{xcolor}
\newcolumntype{.}{D{.}{.}{-1}}
\newcolumntype{d}[1]{D{.}{.}{#1}}
\definecolor{light-gray}{gray}{0.65}
\usepackage{url}
\usepackage{listings}
\usepackage{color}

\definecolor{codegreen}{rgb}{0,0.6,0}
\definecolor{codegray}{rgb}{0.5,0.5,0.5}
\definecolor{codepurple}{rgb}{0.58,0,0.82}
\definecolor{backcolour}{rgb}{0.95,0.95,0.92}

\lstdefinestyle{mystyle}{
	backgroundcolor=\color{backcolour},   
	commentstyle=\color{codegreen},
	keywordstyle=\color{magenta},
	numberstyle=\tiny\color{codegray},
	stringstyle=\color{codepurple},
	basicstyle=\footnotesize,
	breakatwhitespace=false,         
	breaklines=true,                 
	captionpos=b,                    
	keepspaces=true,                 
	numbers=left,                    
	numbersep=5pt,                  
	showspaces=false,                
	showstringspaces=false,
	showtabs=false,                  
	tabsize=2
}
\lstset{style=mystyle}
\newcommand{\Sref}[1]{Section~\ref{#1}}
\newtheorem{hyp}{Hypothesis}

\title{Problem Set 2 - Answers (Applied Stats II)}
\date{18.02. 2026}
\author{Sarah Magdihs}

\begin{document}
	\maketitle
	\section*{Instructions}
	\begin{itemize}
		\item Please show your work! You may lose points by simply writing in the answer. If the problem requires you to execute commands in \texttt{R}, please include the code you used to get your answers. Please also include the \texttt{.R} file that contains your code. If you are not sure if work needs to be shown for a particular problem, please ask.
		\item Your homework should be submitted electronically on GitHub in \texttt{.pdf} form.
		\item This problem set is due before 23:59 on Wednesday February 18, 2026. No late assignments will be accepted.
		
	\end{itemize}
	
	\vspace{.25cm}
	
	\noindent We're interested in what types of international environmental agreements or policies people support (\href{https://www.pnas.org/content/110/34/13763}{Bechtel and Scheve 2013)}. So, we asked 8,500 individuals whether they support a given policy, and for each participant, we vary the (1) number of countries that participate in the international agreement and (2) sanctions for not following the agreement. \\
	
	\noindent Load in the data labeled \texttt{climateSupport.RData} on GitHub, which contains an observational study of 8,500 observations.
	
	\begin{itemize}
		\item
		Response variable: 
		\begin{itemize}
			\item \texttt{choice}: 1 if the individual agreed with the policy; 0 if the individual did not support the policy
		\end{itemize}
		\item
		Explanatory variables: 
		\begin{itemize}
			\item
			\texttt{countries}: Number of participating countries [20 of 192; 80 of 192; 160 of 192]
			\item
			\texttt{sanctions}: Sanctions for missing emission reduction targets [None, 5\%, 15\%, and 20\% of the monthly household costs given 2\% GDP growth]
			
		\end{itemize}
		
	\end{itemize}
	
	\newpage
	\noindent Please answer the following questions:
	
	\begin{enumerate}
		\item
		Remember, we are interested in predicting the likelihood of an individual supporting a policy based on the number of countries participating and the possible sanctions for non-compliance.
		\begin{enumerate}
			\item [] Fit an additive model. Provide the summary output, the global null hypothesis, and $p$-value. Please describe the results and provide a conclusion.
			
		\end{enumerate}
		
		\section*{Task 1: Additive Logistic Regression Model}
		\vspace{.25cm}
		\noindent In order to fulfil the task, I fit an additive model as outlined above. Before this, however, I took a quick look at the data. As can be seen in the code below, I decided to recode the input variables. While I believe that RStudio actually takes the first category as the reference category, I decided to explicitly select the reference category myself (in case my assumption was wrong). To do so, I first unordered the factor variables $countries$ and $sanctions$ since the command $relevel$ does not work on ordered factors. Then I set the reference category to "20 of 192" and "None", respectively. 
		
		\lstinputlisting[language=R, firstline=49, lastline=74]{PS02_answersSM.R}  
		
		\noindent As a next step, I specified the additive model using the $glm$ function. Specifically, I regressed $choice$ on $countries$ and $sanctions$, using $binomial$ as the distribution for the model. 
		
		\lstinputlisting[language=R, firstline=83, lastline=87]{PS02_answersSM.R}  
		\lstinputlisting[language=R, firstline=105, lastline=108]{PS02_answersSM.R}  
		
		\noindent The following table was the result: 
		
		% Table created by stargazer v.5.2.3 by Marek Hlavac, Social Policy Institute. E-mail: marek.hlavac at gmail.com
		% Date and time: Wed, Feb 18, 2026 - 14:53:42
		\begin{table}[H] \centering 
			\caption{Logistic Regression: Support for An Environmental Policy (International)} 
			\label{} 
			\begin{tabular}{@{\extracolsep{5pt}}lc} 
				\\[-1.8ex]\hline 
				\hline \\[-1.8ex] 
				& \multicolumn{1}{c}{\textit{Dependent variable:}} \\ 
				\cline{2-2} 
				\\[-1.8ex] & choice \\ 
				\hline \\[-1.8ex] 
				countries80 of 192 & 0.336$^{***}$ \\ 
				& (0.054) \\ 
				& \\ 
				countries160 of 192 & 0.648$^{***}$ \\ 
				& (0.054) \\ 
				& \\ 
				sanctions5\% & 0.192$^{***}$ \\ 
				& (0.062) \\ 
				& \\ 
				sanctions15\% & $-$0.133$^{**}$ \\ 
				& (0.062) \\ 
				& \\ 
				sanctions20\% & $-$0.304$^{***}$ \\ 
				& (0.062) \\ 
				& \\ 
				Constant & $-$0.273$^{***}$ \\ 
				& (0.054) \\ 
				& \\ 
				\hline \\[-1.8ex] 
				Observations & 8,500 \\ 
				Log Likelihood & $-$5,784.130 \\ 
				Akaike Inf. Crit. & 11,580.260 \\ 
				\hline 
				\hline \\[-1.8ex] 
				\textit{Note:}  & \multicolumn{1}{r}{$^{*}$p$<$0.1; $^{**}$p$<$0.05; $^{***}$p$<$0.01} \\ 
			\end{tabular} 
		\end{table}
		
		\noindent As a next step, I focused on testing the global null hypothesis using $anova$. Specifically, the null hypothesis in this scenario is that all slopes are equal to zero, whereas the alternative hypothesis is that at least one coefficient is unequal to zero. In order to assess this, we need a model to compare the additive model to (=the Null Model): 
		
		\lstinputlisting[language=R, firstline=90, lastline=100]{PS02_answersSM.R}  
		\begin{verbatim}
			Model 1: choice ~ 1
			Model 2: choice ~ countries + sanctions
			Resid. Df Resid. Dev Df Deviance  Pr(>Chi)    
			1      8499      11783                          
			2      8494      11568  5   215.15 < 2.2e-16 ***
			---
			Signif. codes:  0 ‘***’ 0.001 ‘**’ 0.01 ‘*’ 0.05 ‘.’ 0.1 ‘ ’ 1
		\end{verbatim}
		\lstinputlisting[language=R, firstline=110, lastline=113]{PS02_answersSM.R}  
		
		% Table created by stargazer v.5.2.3 by Marek Hlavac, Social Policy Institute. E-mail: marek.hlavac at gmail.com
		% Date and time: Wed, Feb 18, 2026 - 15:01:49
		\begin{table}[!htbp] \centering 
			\caption{Anova: Global Null Hypothesis Testing} 
			\label{} 
			\begin{tabular}{@{\extracolsep{5pt}} cccccc} 
				\\[-1.8ex]\hline 
				\hline \\[-1.8ex] 
				& Resid. Df & Resid. Dev & Df & Deviance & Pr(\textgreater Chi) \\ 
				\hline \\[-1.8ex] 
				1 & $8,499$ & $11,783.410$ & $$ & $$ & $$ \\ 
				2 & $8,494$ & $11,568.260$ & $5$ & $215.150$ & $0$ \\ 
				\hline \\[-1.8ex] 
			\end{tabular} 
		\end{table} 
		
		\noindent \textbf{In Conclusion}: 
		\noindent Firstly, the global null hypothesis test shows that the $H_0$ can be rejected since $p-value <0.05$.  This means that at least one of the coefficients is significantly related affects the outcome variable (=whether an individual supports the policy in question).  
		
		\noindent Secondly, let's take a closer look at the actual output of the logistic regression model.
		\begin{enumerate}
			\item When 20 (out of 192) countries participate and there are no sanction, the expected odds that an individual supports the policy is $exp(-0.273) =0.761$ (baseline odds ratio). This means that, on average, individuals do not support the policy in question. 
			\item Regarding \textbf{countries}: Broadly speaking, an increase in participating countries is associated with an increase in the probability that an individual supports the policy in question. Also, please note that the reference category for the following explanations is "20 out of 192" countries. When 80 countries participate, the log-odds increase by 0.336 (holding everything else constant). In other words, individuals are 1.4 times as likely to support the policy ($exp(0,336)=1.4$). Similarly, the log-odds increase by 0.648 (holding everything else constant). This means that individuals are almost twice as likely to support a policy if most countries participate. 
			\item Regarding \textbf{sanctions}: The relationship between sanctions and choice is a bit more complex. Compared to no sanctions, a 5\% penalty is associated with an 0.192 increase in the log-odds (all else constant). Thus, individuals are roughly 1.21 times as likely to support a policy, when compared to no sanctions. However, higher sanctions are associated with a decrease in the log-odds by 0.133 (15\% sanctions) and 0.304 (20\% sanctions), respectively. Therefore, when compared to the reference category and holding all else constant, individuals are less likely to support a policy in these cases than when there are no sanctions at all. 
			\item \textbf{Overall}, then, this suggest that international participation has a significant, positive effect on the probability that individuals support a given policy, whereas sanctions have a significant but mixed effect. While the model obviously cannot speak to this, I could imagine that the positive effect on participation could be related to the perceived legitimacy of a policy. In comparison, sanctions might be viewed as too strict if they exceed a certain limit. 
		\end{enumerate}
		\item
		If any of the explanatory variables are statistically significant in this model, then:
		\begin{enumerate}
			\item
			For the policy in which nearly all countries participate [160 of 192], how does increasing sanctions from 5\% to 15\% change the odds that an individual will support the policy? (Interpretation of a coefficient)
			\item
			For the policy in which very few countries participate [20 of 192], how does increasing sanctions from 5\% to 15\% change the odds that an individual will support the policy? (Interpretation of a coefficient)
			\item
			What is the estimated probability that an individual will support a policy if there are 80 of 192 countries participating with no sanctions? 
			
			\section*{Answer to Task 2:} 
			\vspace{.25cm}
			\noindent Fortunately, Task 2a and 2b can be answered simultaneously. This is because the odds ratio actually does not change here. Since we are working with a additive model, we actually assume that the effect does not depend on the number of countries that participate. 
			
			I calculated how increasing the sanctions from 5\% to 15\% changes the odds that an individual will support the policy by 'isolating' the relevant coefficients and then exponentiation their difference. Lastly, I also calculated the "decrease":
			
			\lstinputlisting[language=R, firstline=120, lastline=137]{PS02_answersSM.R}  
			
			\noindent As a result, RStudio provided these outputs: 
			
			\begin{verbatim}
				> odds_ratio
				sanctions15% 
				0.7224531 
				> decrease
				sanctions15% 
				0.2775469 
			\end{verbatim} 
			
			\noindent This means that increasing sanctions from 5\% to 15\% decreases the odds of an individual supporting the policy by about 27.7\%.
			
			\noindent Moving on to Task 2c, I estimatde the probability that an individual will support the policy when 80 of 192 countries participate and there are no sanctions by using the $predict$ function:
			
			\lstinputlisting[language=R, firstline=139, lastline=143]{PS02_answersSM.R}  
			
			\noindent As a result, RStudio provided this output: 
			
			\begin{verbatim}
				> est_prob_80
				1 
				0.5159191 
			\end{verbatim} 
			
			\noindent Hence, the estimated probability is roughly 51.6\%. 
			
		\end{enumerate}
		\item
		Would the answers to 2a and 2b potentially change if we included an interaction term in this model? Why? 
		\begin{itemize}
			\item Perform a test to see if including an interaction is appropriate.
		\end{itemize}
	\end{enumerate}
	
	\section*{Answer to Task 3:} 
	\vspace{.25cm}
	\noindent Yes, the answer to question 2a and 2b could potentially change, if we were to include an interaction term. Unlike an additive model (which assumes that the effect of sanctions is not dependent on countries), a model with interaction terms would expect that the effect of sanctions depends on the number of participating countries. 
	
	\noindent To be sure, however, whether an interaction term would actually change the results, we would have to test whether the interaction actually improves our model. To do this, we can compare both models using $anova$. 
	
	\lstinputlisting[language=R, firstline=149, lastline=170]{PS02_answersSM.R}  
	
	\noindent Using the code above, R provides us with the following output:
	
	% Table created by stargazer v.5.2.3 by Marek Hlavac, Social Policy Institute. E-mail: marek.hlavac at gmail.com
	% Date and time: Wed, Feb 18, 2026 - 17:02:58
	\begin{table}[H] \centering 
		\caption{Anova: Additive vs. Interaction} 
		\label{} 
		\begin{tabular}{@{\extracolsep{5pt}} cccccc} 
			\\[-1.8ex]\hline 
			\hline \\[-1.8ex] 
			& Resid. Df & Resid. Dev & Df & Deviance & Pr(\textgreater Chi) \\ 
			\hline \\[-1.8ex] 
			1 & $8,494$ & $11,568.260$ & $$ & $$ & $$ \\ 
			2 & $8,488$ & $11,561.970$ & $6$ & $6.293$ & $0.391$ \\ 
			\hline \\[-1.8ex] 
		\end{tabular} 
	\end{table}
	
	\noindent Since the p-value is $>0.05$, we fail to reject the null hypothesis. This means that the inclusion of an interaction term does not improve our model; the additive model is sufficient. Consequently, the answers to 2a and 2b should remain the same. 
	
	
\end{document}
